\documentclass{article}
\usepackage{amsmath, amsthm}        % 数学公式和定理环境
\usepackage{xcolor}                % 颜色支持
\usepackage{tcolorbox}             % 方框宏包
\tcbuselibrary{theorems, skins}    % 加载定理和皮肤库

% 定义红色方框环境(Definition)
\newtcbtheorem[number within=section]{definition}{Definition}{
  colback=red!5!white,    % 背景颜色
  colframe=red!75!black,  % 边框颜色
  fonttitle=\bfseries,    % 标题字体
  sharp corners,          % 直角边框
  title style={fill=red!20!white}, % 标题背景颜色
  boxrule=1pt,            % 边框粗细
  arc=0mm,                % 直角边角
  left=2mm, right=2mm,    % 左右边距
}{def}

% 定义绿色方框环境(Theorem)
\newtcbtheorem[number within=section]{theorem}{Theorem}{
  colback=green!5!white,
  colframe=green!50!black,
  fonttitle=\bfseries,
  sharp corners,
  title style={fill=green!20!white},
  boxrule=1pt,
  arc=0mm,
  left=2mm, right=2mm,
}{thm}

\begin{document}

\section{Group Theory}

% 使用红色方框定义环境
\begin{definition}{Inner automorphisms}{inner_automorphism}
Let \(G\) be a group, and let \(a \in G\). The function \(\phi_{a}\) defined by
\[
\phi_{a}(x) = a x a^{-1} \quad \text{for all } x \in G
\]
is called the \textbf{inner automorphism} of \(G\) induced by \(a\).
\end{definition}

% 使用绿色方框定理环境
\begin{theorem}{Automorphism Groups}{automorphism_group}
The set of automorphisms of a group (\(\operatorname{Aut}(G)\)) and the set of inner automorphisms (\(\operatorname{Inn}(G)\)) are both groups under the operation of function composition.
\end{theorem}

\end{document}