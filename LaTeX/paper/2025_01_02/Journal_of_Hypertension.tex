\documentclass[12pt]{article}

% 导入中文支持的包
\usepackage[UTF8]{ctex} % ctex 包支持中文处理

% 设置页面布局(可选)
\usepackage[a4paper,margin=1in]{geometry}

% 导入常用包(根据需要添加)
\usepackage{amsmath, amssymb} % 数学公式
\usepackage{booktabs} % 提供高级表格线
\usepackage{multirow} % 单元格合并
\usepackage{array} % 表格排版控制
\usepackage{graphicx} % 插入图片
\usepackage{hyperref} % 超链接
\hypersetup{
    colorlinks=true, % 激活链接颜色
    linkcolor=blue, % 文档内部链接颜色,适用于图、表在文中的引用
    citecolor=blue, % 文献引用链接颜色,适用于文中参考文献引用
    urlcolor=blue  % 外部URL链接颜色,适用于参考文献添加超链接
}
\usepackage{xcolor} % 颜色支持

% 标题和作者信息(可选)
\title{颅内压升高时脑血流动力学原理:来自外周循环的教训}
\author{王成}
\date{\today}

\begin{document}
\maketitle
\newpage

\tableofcontents
\newpage

\section{摘要}
\subsection{背景}
大脑具有高度丰富的血管和血流量,并依赖与持续的血流以维持正常的生理功能。
虽然大脑局限于颅骨内,但在正常情况下,颅内的压力通常小于15mmHg,并显示出与动脉脉搏相关的小脉动。
颅内动脉搏动性血流动力学此前已被研究过,但仍然不足以解释,尤其是在头部受伤后颅内压(ICP)变化时。

\subsection{方法}
在寻求连贯的解释时,我们使用高精度压力计系统侵入性地测量颅内压 (ICP) 和桡动脉压力 (RAP),同时利用跨颅多普勒测量大脑中动脉流速 (MCAFV)、以及使用广义传递函数技术从 R​​AP 生成的中心主动脉压 (CAP),研究了八名闭合性头部创伤后年轻无意识、接受呼吸机治疗的成年患者。
我们着重研究了自发性颅内压升高(“平台波”)的血管效应。

\subsection{结果}
颅内压(ICP)平均值从29 mmHg上升到53 mmHg并未引起颅骨外压力或心率的持续变化,但ICP脉动波幅从8 mmHg增加到20 mmHg,ICP波形开始类似于主动脉中的波形。
脑灌注压(即中央主动脉压–ICP),与跨壁压相当,从61 mmHg下降到36 mmHg。大脑平均中动脉血流速度(MCAFV)从53 cm/s下降到40 cm/s,而脉动性MCAFV从77 cm/s增加到98 cm/s。这些显著变化(所有P < 0.01)可用Monro-Kellie学说来解释,因为大脑受到了压迫,就像外部压力作用于肢体时的情况。

\subsection{结论}
研究结果强调了在颅内压升高时降低其水平的重要性,以及减少来自下半身的波反射的额外好处。

\section{引言}

\section{材料和方法}



\end{document}