\documentclass[12pt]{article}
\usepackage{amsmath}
\usepackage{booktabs}
\usepackage{geometry}
\geometry{a4paper, margin=1in}
\usepackage{ctex} % 支持中文
\setCJKmainfont{SimSun} % 设置中文字体,例如宋体
\usepackage{natbib} % 用于引用

\title{基于粒子群优化算法的磁悬浮离心式血泵参数优化设计}
\author{}
\date{March 2025}

\begin{document}

\maketitle

\begin{abstract}
为了解决磁悬浮离心式血泵叶轮的优化问题,提出了一种优化方法。
分析了叶轮参数的敏感性,然后建立了参数与流量和溶血率之间的近似模型,并通过粒子群优化算法找到了最佳参数。
通过仿真验证了优化方法的有效性。与原始血泵相比,流量增加了27.9\%,溶血率降低了78.6\%。
\end{abstract}

\section{引言}
中风是一种发病率和死亡率都很高的脑血管疾病。对于中风,低温输血是一种有效的治疗方法,磁悬浮心脏血泵起着至关重要的作用。
在本研究中,以自主研发的小体积离心血泵为研究对象。在确保足够流量的同时,降低溶血水平是离心血泵优化的主要关注点。
在不增加叶轮半径的情况下,提出了一种优化方法,以提高流量并降低溶血率。

\section{血泵参数的敏感性分析}
血泵的参数如图1所示,包括叶轮形式、叶片形状以及叶片尖端间隙、回流孔和回流通道间隙的尺寸。这些参数与流量和溶血率之间的敏感性分析如图2所示。

\section{构建近似模型并搜索优化}
根据图2,叶片形状对流量和溶血率的影响最大,而叶片形状由螺旋角控制。因此,螺旋角、流量和溶血率之间的关系被拟合为一个函数,如图3所示。然后,使用PSO算法找到在满足溶血率的情况下最大化流量的螺旋角,得到的最佳螺旋角为$72.9^\circ$。

\section{仿真验证}
上述最优解的设计螺旋角进行了重新建模,然后进行了CFD仿真验证,如图4所示。拟合模型的预测值和CFD计算结果如表\ref{tab:comparison}所示。流量差异为1.6\%,溶血率差异为2.6\%。

\begin{table}[h]
    \centering
    \caption{Comparison Between Fitting Model and CFD Calculation}
    \label{tab:comparison}
    \begin{tabular}{lcc}
        \toprule
        Item & Flow rate (ml/min) & Hemolysis rate \\
        \midrule
        Predicted value & 271.0 & $1.19 \times 10^{-4}$ \\
        CFD value & 275.6 & $1.16 \times 10^{-4}$ \\
        Previous value & 215.5 & $5.42 \times 10^{-4}$ \\
        \bottomrule
    \end{tabular}
\end{table}

\section{结论}
仿真结果验证了优化方法的准确性和优势。与原始泵相比,流量增加了27.9\%,溶血率降低了78.6\%。



\end{document}