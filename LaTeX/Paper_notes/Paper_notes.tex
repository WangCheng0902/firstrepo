\documentclass[titlepage,12pt]{article}
\usepackage[UTF8]{ctex}
\usepackage{xeCJK}
\usepackage{graphicx}
\usepackage{lettrine}
\usepackage{shapepar}
\usepackage{enumitem}
\usepackage{listings}
\usepackage{geometry}
\usepackage{amsmath}
\geometry{a4paper,left=3cm,right=3cm}

\title{Questions and Notes}
\author{wang cheng}
\date{Dec 2024}

\begin{document}

\maketitle

\section{笔记}
\subsection{Intracranial pressure and  cerebral blood flow}
\subsubsection{摘要}
颅内压由脑组织、脑脊液和脑血容量三者的体积决定,任一内容物的变化都需要其他内容物来补偿变化内容物的体积,
否则颅内压会发生变化。颅内压的相对恒定对于维持脑灌注压处于正常是至关重要的,正常的脑灌注压使得脑血流在全局和局部得到调节,
以防止系统性动脉血压发生变化引起的脑灌注过度或不足,并满足不同脑区域动态氧气和底物需求。
\subsubsection{脑组织}
脑实质大约有1400克,由神经元、胶质细胞和细胞外液组成。血脑屏障(BBB)由毛细血管内皮细胞之间的紧密连接构成,
将血液与脑间质液分隔开来,以为神经元活动提供适宜的环境。大脑组织病理性增加的原因包括肿瘤、细胞毒性水肿(由于细胞膜破裂)和血管源性水肿(由于BBB破坏)。
\subsubsection{脑脊液}
脑脊液位于蛛网膜和软脑膜之间。其功能包括维持稳定的化学环境,支持代谢产物和神经传递物质的运输,并提供葡萄糖。可以说大脑漂浮在脑脊液中,
脑受线性和剪切机械力的影响也通过脑脊液的保护性位移而减少。
\subsubsection{脑血容量}
脑的动脉血供是由颈内动脉和椎动脉,分别通过颈动脉管和枕大孔进入颅内。血液通过脑静脉、静脉窦和颈内静脉排出,通过颈静脉孔从颅腔排出。
正常颅内血容量约为150毫升,其中三分之二位于静脉系统中。
\subsubsection{正常与异常颅内压}
成年人的正常颅内压为5至13毫米汞柱,由于动脉脉压和呼吸引起的轻微周期性变化。
超过20mmHg的ICP会导致局部缺血区,而全脑缺血发生在超过50mmHg的情况下。

除了降低全球脑灌注压外,导致颅内压升高的病变的“mass effect”也会导致脑组织在颅内固定结构上发生局部位移,这被称为疝。
颏叶(经幕下)疝涉及内侧颞叶经过幕突下移,可能导致第三对脑神经麻痹、后大脑动脉受压和半身瘫痪。
小脑扁桃体疝(或“圆锥”)是颅内压升高的潜在致命后果,小脑扁桃体经过枕骨大孔下移,压迫脑干中的重要呼吸中枢。
\subsubsection{脑血流和自动调节}
大脑依赖葡萄糖的氧化磷酸化作为其主要能量来源,因此对缺氧和低血糖非常不耐受。
因此保持稳定持续的脑血流是至关重要的。

脑灌注压(CPP)由平均动脉压(MAP)和颅内压(ICP)确定。应严格考虑有效静脉压(或颈静脉球压力),但通常被认为在临床上不重要。
尽管动脉压力(MAP)存在变异,因此脑灌注压(CPP)也不同,但脑血流(CBF)通过一种被称为自动调节的机制维持在大约每100g组织/min 50毫升的恒定速率。在正常情况下,脑血管的自动调节能够防止在大约100mmHg CPP范围内的CBF变化。

这种严格的控制主要通过脑动脉小动脉的肌源性反应实现,血管平滑肌在对CPP增加的情况下收缩血管,并在压力下降的情况下扩张。脑阻力血管的这种固有特性,被称为Bayliss效应,独立于压力来控制下游灌注,并限制了毛细血管受到的管腔内压力,毛细血管壁薄易受损。
\section{The complexity of intracranial pressure as an indicator of cerebral autoregulation}
\subsection{摘要}
颅内压是当今用于指导重症监护病房(ICU)中急性神经患者的主要神经监测指标之一。本文评估了颅内压增高期间的复杂性,并将其与颅内张力稳定期进行比较。
使用贝丝以色列代康内斯医疗中心的多参数智能监护III(MIMIC-III)数据库,评估了颅内张力稳定期和高颅内压期的复杂度,使用两个量化器:置换熵和各自的缺失模式数量。
两者都表明高血压信号中的复杂度丧失。通过使用脑自我调节和脑血流动力学的动力模型来给出了这种复杂度丧失的生理解释。
\subsubsection{简介}
颅内压(ICP)是当今用于指导重症监护室(ICU)中的急性神经病患治疗的主要神经监护指标之一。它由颅腔及其内容(脑组织、脑脊液和血容量)的关系决定,并受复杂机制调控,这种机制允许在不同情况下保持其数值。
这一机制的一个组成部分是脑自主调节(CA),它使得在血压变化时,能够对血流和容积进行变化。CA和ICP之间有复杂的相互关系,其中ICP的维持取决于CA的保护,而CA的维持又取决于ICP,因为颅内高压(ICH)的存在耗尽CA机制。
许多数学模型被提出来理解脑自动调节的动力学[3-5]。从系统动力学的角度出发,根据Ursino等人提出的模型[5]描述了颅内压与脑血液动力学之间的相互作用,存在保持脑自动调节的三种不同的负反馈控制回路,以及由于动脉-小动脉血容量主动变化而导致的正反馈回路引起的不稳定性。
由反馈IV引起的CA不稳定性是本文使用复杂性框架研究的不稳定性类型:如果患者具有适度的脑脊液(CSF)排出阻力和低颅内顺应性,则这将触发一个正反馈循环,随着颅内顺应性恶化,正反馈循环在颅内动力学中变得更具影响力,使系统远离最佳平衡,类似于罗斯纳的扩血管级联。
当脑灌注压下降时,还会导致脑血流量的减少,随之产生血管扩张效应以维持恒定流量,导致脑血容量增加,从而增加ICP,进而引起脑灌注压更大程度地减小,导致恶性循环,即正反馈。其他三个反馈逃逸通路试图将ICP保持在正常值。

\section{颅内压-血泵耦合模型的可行性分析}
\subsection{Fontan循环腔肺辅助装置的血流动力学自适应控制及生理特征评估}
文章中对CPAD模型使用表示血泵转速变化率的微分方程进行描述,公式如下:
\begin{equation*}
    J \frac{d\omega}{dt} = \frac{3}{2} K_B I - B \omega - a_0 \omega^3 - a_1 F_P \omega^2
\end{equation*}
上式中的各种参数通过实验确定,CPAD 的血泵转速用 $\omega$ 表示,单位为 rad/s,是随着控制变量 I 变化的因变量;$F_P$ 为血泵流量,单位为 ml/s。
CPAD 的血泵流量由以下微分方程描述:
\begin{equation*}
    \frac{dF_P}{dt} = -\frac{b_0}{b_1} F_P - \frac{b_2}{b_1} \omega^2 + \frac{1}{b_1} CPPH
\end{equation*}
上式中的参数$b0$、$b1$和$b2$通过实验确定。

本论文中的CPPH实际值可通过两种方式获得。一种方式是假设肺动脉压力和肺静脉压力可以通过压力传感器直接测量,通过它们的差值得到CPPH,所对应的  生理控制算法被称为基于传感器的生理控制算法(sensor-based);另一种更接近实际应用的方式是采用扩展型卡尔曼滤波器(EKF)[91, 92]和格雷-赛维斯基滤波器(GSF)[93, 94] ,  通过带噪声的CPAD固有血泵转速对CPPH的实际值进行估算,得到的是CPPH的估算值 (CPPHe),即无传感器生理控制算法(sensorless),无需使用不可靠的额外压力传感器,因此这种方法更具现实意义。

\subsection{心脏泵多目标控制策略及系统设计}
该论文根据心脏泵的水力特性,心脏泵效率公式和Choi等提出的模型,考虑流量压差之间的水力损失,构建了新的泵模型,如下:
\begin{equation*}
    P_p = \frac{\beta_0 \omega}{\omega_0} \cdot Q - k(Q - Q_{opt} \cdot \frac{\omega^2}{\omega_0}) + b \cdot \left(\frac{\omega}{\omega_0}\right)^2 + \frac{\beta_1 \omega}{\omega_0} \cdot \frac{dQ}{dt}
\end{equation*}
可改写为:
\begin{equation*}
    P_p = R_p Q + L_p \frac{dQ}{dt} + \beta_2 \omega^2
\end{equation*}
其中,\(R_p = \frac{\beta_0 \omega}{\omega_0} - kQ + \frac{2k\beta_0 \omega Q_{opt}}{\omega_0}\),\(L_p = \frac{\beta_1 \omega}{\omega_0}\),\(\beta_2 = \frac{-kQ_{opt} + b}{\omega_0^2}\)。
\(Q_{opt}\) 为最佳工况点的数据源值,\(\omega\) 为对应给定中为模拟各种情形调整的任意参数,\(Q\) 为该参数下的泵流量,\(k\),\(b\),\(\beta_0\),\(\beta_1\) 均为模型参数系数。

\subsection{心室辅助装置控制系统的建模、设计与实现}
从轴流泵流体力学特征中可以看出,它的性质可以用压差 $H$,泵流 $Q$ 和转速 $\omega$ 来表达。稳态下轴流泵特性表明:在某一转速下,泵进出口压差可表示为与泵  流动相关二阶或者三阶多项式。但进一步研究发现泵流与转速平方之间存在线性关系,其模型表达式如以下公式:
\begin{equation*}
    H = \Delta P = P_o - P_i = b_0 Q + b_1 Q^2 + b_2 \omega^2
\end{equation*}
其中 $Pi$ 与 $Po$ 分别代表泵的入口、出口压强,$b0$、$b1$、$b2$ 三个系数于泵的结构有关,需要通过实验测定。

\subsection{脉动流左心室辅助装置血流动力学及生理控制研究}
为了模拟植入 LVAD 的心血管系统血液动力学情况,需要建立 LVAD 在工作时的扬程-流量-转速模型。Kitamura 等人通过对轴流式 LVAD 进行研究,  结合轴流式 LVAD 的水力特性,得到了离心式 LVAD 的模型如公式所示:
\begin{equation*}
    H = \beta \omega^2 + R_{p2} Q + R_w Q^2 + L_p \frac{dQ}{dt}
\end{equation*}

式中: $H$ 为 LVAD 的扬程,单位 mmHg;$Q$ 为泵出流量,单位 L/min;$\frac{dQ}{dt}$ 为输出流量 $Q$ 的变化率;$\omega$为叶轮转速,单位 rpm;$β$、 $R_p$ 、 $R_{\omega}$ 和 $L_p$ 为常数,取值分别为 $1.5674×10^-5(mmHg/rpm^2)$、$-0.1169(mmHg·s·mL^-1)$、$-0.0023(mmHg·s^-2 mL^-2)$和$-0.0034(mmHg·s^2 mL^-1)$。


\subsection{离心血泵控制策略及体外模拟循环系统研究}
根据 Choi 等人的研究,可以得到离心血泵的微分方程如下:
\begin{equation*}
    \frac{dQ_x}{dt} = -\frac{\beta_0}{\beta_1} Q_x - \frac{\beta_2}{\beta_1} \omega^2 + \frac{1}{\beta_1} H
\end{equation*}
其中$Q_x$表示血泵流量,$\frac{dQ_x}{dt}$表示流量的变化率,$H$表示血泵进出口的压力差,$\omega$表示血泵叶轮转速。$\beta_0$ 、$\beta_1$、 $\beta_2$为实验常数。

\end{document}
