\documentclass[12pt,a4paper]{article} % 使用article文档类,字体大小为12pt,纸张为A4

% 引入必要的包
\usepackage[UTF8]{ctex} % 支持中文
\usepackage{amsmath} % 数学公式
\usepackage{amssymb} % 数学符号
\usepackage{geometry} % 设置页面边距
\usepackage{graphicx} % 插入图片
\usepackage{hyperref} % 超链接
\usepackage{listings} % 插入代码
\usepackage{xcolor} % 自定义颜色
\usepackage{booktabs} % 表格优化
\usepackage{enumitem} % 调整列表格式

% 页面设置
\geometry{left=2.5cm,right=2.5cm,top=2.5cm,bottom=2.5cm} % 设置页边距

% 代码样式设置
\lstset{
    basicstyle=\ttfamily\small, % 设置代码字体和大小
    numbers=left, % 显示行号
    numberstyle=\tiny, % 行号字体
    keywordstyle=\color{blue}, % 关键词颜色
    commentstyle=\color{gray}, % 注释颜色
    stringstyle=\color{red}, % 字符串颜色
    breaklines=true, % 自动换行
    frame=single, % 添加边框
    backgroundcolor=\color{gray!10} % 背景颜色
}

\title{随手记} % 文档标题
\author{王成} % 作者
\date{\today} % 日期

\begin{document}

\maketitle % 生成标题

\section{知乎}
\subsection{集总参数模型中的地解释}
\colorbox{yellow!20}{
    \parbox{\dimexpr\linewidth-2\fboxsep}{
        由于血液从心脏射出,流经动脉,然后进入毛细血管和静脉系统,流回到心脏。在电网络模型中,可以将心脏视为交流电源。由于静脉系统的血压较低,可以将静脉视为零电位,血液流入静脉视为电路中的“接地”。动脉系统的整体电网络模型如图所示:
    }
}

\section{叙述性内容}
\subsection{本课题的研究内容及意义}
本课题主要针对缺血性脑卒中患者的治疗,通常在患者中风后被送往医院的4.5\textasciitilde6小时内,先通过CT,MRI等影像技术对血管狭窄或阻塞部位进行快速定位,然后基于该部位制定合适的手术如动脉机械取栓(EVT)或者静脉溶栓(IVT)等。
由于缺血性卒中的治疗具有时间敏感性,延迟会显著影响预后,如脑组织和神经系统因缺血缺氧而死亡凋零,致使患者残疾或死亡,于是在患者通过手术血管再通后,需要辅以自体血低温脑灌注,以降低脑部新陈代谢速率,减少脑细胞死亡速率,延长治疗时间窗,挽救缺血半暗带,降低患者致死致残率。

\subsection{子标题2}
插入公式示例:
\begin{equation}
    E = mc^2
\end{equation}



插入代码示例:
\begin{lstlisting}[language=Python, caption=Python代码示例]
def hello_world():
    print("Hello, World!")
\end{lstlisting}

\section{结论}
这里是结论部分。

\end{document}